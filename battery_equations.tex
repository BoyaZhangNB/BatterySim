\documentclass[12pt]{article}
\usepackage{amsmath, amssymb, amsfonts}
\usepackage{geometry}
\geometry{margin=1in}

\title{Complete Battery Charging Model}
\author{}
\date{}

\begin{document}
\maketitle

\section*{State Vector}

The simulation uses the packed state vector
\[
y = \begin{bmatrix}
V \\ I \\ R \\ T \\ SOC \\ SEI
\end{bmatrix},
\]
where:
\[
\begin{aligned}
V &: \text{terminal/OCV voltage (V)}, \\
I &: \text{current into the battery (A)}, \\
R &: \text{internal resistance (}\Omega\text{)}, \\
T &: \text{temperature (K)}, \\
SOC &: \text{state of charge (0--1)}, \\
SEI &: \text{SEI layer thickness (arb. units)}.
\end{aligned}
\]

The full ODE system is:
\[
\frac{dy}{dt} = \sum_{\text{mechanisms}} \frac{dy}{dt}^{(\text{mech})}.
\]

% =====================================================
\section{Charging Model (SOC Dynamics)}
\subsection*{Differential Equation}

\[
\frac{dSOC}{dt} = \frac{I}{C_{\text{nominal}} \cdot 3600}
\]

All other partial derivatives from this mechanism are zero.

\subsection*{Parameters}
\[
\begin{aligned}
C_{\text{nominal}} &: \text{nominal capacity (Ah)}, \\
3600 &: \text{seconds per hour}.
\end{aligned}
\]

% =====================================================
\section{Thermal Model}

\subsection{Ohmic Heating}
\[
\left(\frac{dT}{dt}\right)_{\text{ohmic}} 
= \frac{I^2 R}{m c}
\]

\subsection{Overpotential Heating}
\[
\left(\frac{dT}{dt}\right)_{\text{overpotential}}
= \frac{I\,(V_{\text{source}} - V)}{m c}
\]

\subsection{Cooling (Newton's Law)}
\[
\left(\frac{dT}{dt}\right)_{\text{cooling}}
= -k(T - T_{\text{amb}})
\]

\subsection{Total Temperature ODE}
\[
\frac{dT}{dt} =
\frac{I^2 R}{m c}
+ \frac{I\,(V_{\text{source}} - V)}{m c}
- k(T - T_{\text{amb}})
\]

\subsection*{Parameters}
\[
\begin{aligned}
m &: \text{mass (kg)}, \\
c &: \text{specific heat (J/(kg K))}, \\
k &: \text{cooling constant (1/s)}, \\
T_{\text{amb}} &: \text{ambient temperature (K)}.
\end{aligned}
\]

% =====================================================
\section{SEI Growth Model (Full Version)}

\subsection{Arrhenius Temperature Factor}
\[
f_T(T) = \exp\left(-\frac{E_a}{R_{\text{gas}} T}\right)
\]

\subsection{Stress Function}
Let the effective anode potential be:
\[
U = \begin{cases}
U_{\text{ocv}}, & \text{if supplied} \\
U_{\text{ref}} \cdot SOC, & \text{otherwise}
\end{cases}
\]

Compute:
\[
\text{exponent} = \gamma_{\text{soc}} (U_{\text{ref}} - U)
\]
\[
\text{current\_term} = \left(1 + \beta_I |I|\right)^{\nu}
\]

Stress multiplier:
\[
f_{QI}(SOC,I,U) =
A \exp\left(\gamma_{\text{soc}} (U_{\text{ref}} - U)\right)
\left(1 + \beta_I |I|\right)^{\nu}
\]

\subsection{SEI Growth ODE}
\[
k_T = k_0 \exp\left(-\frac{E_a}{R_{\text{gas}}T}\right)
\]

\[
\frac{dSEI}{dt} = k_T \, f_{QI}(SOC,I,U)
\]

\subsection*{Parameters}
\[
\begin{aligned}
k_0 &: \text{SEI rate prefactor (m/s or a.u.)}, \\
E_a &: \text{activation energy (J/mol)}, \\
R_{\text{gas}} &: 8.314 \text{ (J/mol K)}, \\
A &: \text{scaling factor}, \\
\gamma_{\text{soc}} &: \text{SOC/potential sensitivity}, \\
\beta_I &: \text{current influence factor}, \\
\nu &: \text{current exponent}, \\
U_{\text{ref}} &: \text{reference potential (V)}.
\end{aligned}
\]

% =====================================================
\section{Simplified SEI Model}

\subsection{Stress Function}
\[
\text{soc\_factor} = \max(0,\min(1,SOC))
\]
\[
\text{current\_factor} = |I|
\]
\[
f_{QI}(SOC,I) = (1 + 2\,\text{soc\_factor}^2)(1 + 0.1|I|)
\]

\subsection{SEI ODE}
\[
\frac{dSEI}{dt} = k_0 \exp\left( -\frac{E_a}{R_{\text{gas}}T} \right)
(1 + 2 SOC^2)(1 + 0.1|I|)
\]

% =====================================================
\section{Algebraic State Updates}

These are applied each timestep before integration.

\subsection{Current Update}
\[
I_{\text{new}}
= \frac{V_{\text{source}} - V}{R}
\]

\subsection{Temperature-Dependent Resistance Update}
\[
R_{\text{new}}
= R_0 \exp\left(
\frac{E_a}{k}\left(
\frac{1}{T} - \frac{1}{T_{\text{ref}}}
\right)
\right)
\]

Parameters:
\[
\begin{aligned}
R_0 &: \text{initial resistance (}\Omega\text{)}, \\
E_a &: 0.7 \text{ (eV)}, \\
k &: 8.314 \text{ (eV/K)}, \\
T_{\text{ref}} &: 298 \text{ (K)}.
\end{aligned}
\]

\subsection{OCV From SOC (Table Interpolation)}
For SOC (0--100\%):

Given pairs $(s_i, V_i)$,
\[
OCV(s) =
V_{\text{low}}
+
\frac{s - s_{\text{low}}}{s_{\text{high}} - s_{\text{low}}}
\left(V_{\text{high}} - V_{\text{low}}\right)
\]

\subsection{Updated State}
\[
y_{\text{new}}
=
\begin{bmatrix}
OCV(100 \cdot SOC) \\
\frac{V_{\text{source}} - V}{R} \\
R_{\text{new}} \\
T \\
SOC \\
SEI
\end{bmatrix}
\]

% =====================================================
\section{Charging Policies}

\subsection{Constant Voltage (CV)}
\[
V_{\text{source}} = V_{\text{set}}
\]

\subsection{Constant Current (CC)}
\[
V_{\text{source}} = \frac{I_{\text{set}}}{R}
\]

\subsection{Pulse Charging}
Let
\[
t_{\text{cycle}} = t_{\text{on}} + t_{\text{off}}.
\]
Then:
\[
V_{\text{source}}(t)
=
\begin{cases}
\frac{I_{\text{set}}}{R}, & (t \bmod t_{\text{cycle}}) < t_{\text{on}} \\[6pt]
0, & \text{otherwise}
\end{cases}
\]

\subsection{Sinusoidal Charging}
\[
V_{\text{source}}(t)
= \frac{I_{\text{amp}} \sin(2\pi f t)}{R}
\]

% =====================================================
\section{Complete System ODE (As Used in main.py)}

With active mechanisms = \{Thermo, Charging\}:
\[
\frac{dV}{dt} = 0,\qquad
\frac{dI}{dt} = 0,\qquad
\frac{dR}{dt} = 0,
\]
\[
\frac{dSOC}{dt} = \frac{I}{C_{\text{nominal}} \cdot 3600},
\]
\[
\frac{dT}{dt} =
\frac{I^2 R}{m c}
+ \frac{I\,(V_{\text{source}} - V)}{m c}
- k(T - T_{\text{amb}}),
\]
\[
\frac{dSEI}{dt} = 0 \quad(\text{unless SEI model is added}).
\]

\end{document}
