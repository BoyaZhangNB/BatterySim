\documentclass[12pt]{article}

\usepackage{amsmath, amssymb, amsfonts}
\usepackage{geometry}
\usepackage{graphicx}
\usepackage{booktabs}
\usepackage{hyperref}
\geometry{margin=1in}

\title{A Mathematical Model for Lithium-Ion Battery Charging \\[4pt]
\large Comparing Fast-Charging Methods via Coupled ODE Mechanisms}
\author{%
  Anna Cai \and
  Boya Zhang \and
  Dazhou Huang \and
  Gautam Manerikar}
\date{\today}

\begin{document}
\maketitle

\begin{abstract}
Lithium-ion batteries are central to modern energy systems, powering electric vehicles, grid-scale storage, and portable electronics. As demand grows, optimizing charging strategies becomes crucial for improving charging speed, efficiency, and lifetime. In this project we develop a mechanistic, ODE-based model of a single lithium-ion cell undergoing charging, and couple it to several charging control policies (constant voltage, constant current, pulsed charging, and sinusoidal charging). The model incorporates three key mechanisms: (i) thermal dynamics via resistive and overpotential heating with cooling to the environment, (ii) state-of-charge (SoC) evolution under different current profiles, and (iii) long-term degradation through the growth of the solid--electrolyte interphase (SEI) layer on the anode.

Our implementation uses a modular architecture in Python: each physical mechanism contributes its own component of the state derivative, while charging policies specify the applied source voltage. A fourth-order Runge--Kutta (RK4) integrator advances the system in time, with an additional algebraic update layer linking SoC to open-circuit voltage and temperature to resistance drift. We use this framework to compare charging methods in terms of charging speed, thermal stress, and degradation rate, and to identify trade-offs between fast charging and battery lifespan. The structure of our report and modelling approach is inspired by the example fire spread model project from previous iterations of MAT292, adapted from wildland fire dynamics to electrochemical battery behaviour.:contentReference[oaicite:0]{index=0}

\end{abstract}

\tableofcontents

\section{Introduction and Background}

Batteries have become a cornerstone of modern society, powering consumer electronics, electric vehicles (EVs), and renewable energy storage systems. Global EV battery demand and grid-scale storage capacity are increasing rapidly, while battery costs have fallen sharply over the past decade.:contentReference[oaicite:1]{index=1} These trends highlight the importance of accurate battery models to inform and optimize charging strategies.

Intentional charging strategies can extend battery lifetime by reducing stress factors such as high temperature, large overpotentials, and aggressive charge cycles. Studies suggest that carefully designed charging profiles can extend lifetime by up to 30\% while improving fast-charging performance.:contentReference[oaicite:2]{index=2} At the same time, experimental fast-charging protocols (e.g., 10--80\% SoC in under 20 minutes) motivate the need for models that capture both transient dynamics and degradation mechanisms.

Our project fits the MAT292 project description: we build and analyze a real-world ODE-based model, develop computational tools, and aim for a mathematically rigorous final report.:contentReference[oaicite:3]{index=3} The project combines:
\begin{itemize}
  \item physical modelling of battery thermal and electrochemical processes,
  \item numerical solution of coupled nonlinear ODEs,
  \item comparison of control policies (charging protocols),
  \item and analysis of long-term degradation through SEI growth.
\end{itemize}

\subsection{Project Goal}

The overarching goal of this project is:
\begin{quote}
  \emph{To construct and analyze a mechanistic ODE model of lithium-ion battery charging that can compare the speed, efficiency, and degradation impacts of multiple charging strategies.}
\end{quote}

Specifically, we want to answer:
\begin{enumerate}
  \item How do constant-voltage, constant-current, pulsed, and sinusoidal charging methods differ in:
  \begin{itemize}
    \item time to charge from 0\% to 100\% SoC,
    \item total heat generated during charging,
    \item and growth rate of the SEI layer?
  \end{itemize}
  \item Under what parameter regimes do thermal or SEI effects become limiting factors for fast charging?
  \item How might model insights guide recommendations for practical charging strategies?
\end{enumerate}

These modelling questions parallel the structure of the fire spread project, which combined several submodels (elliptical spread, spotting, probability of ignition) to answer a high-level probabilistic question about fire behaviour.:contentReference[oaicite:4]{index=4} Here, we similarly combine thermal, electrochemical, and degradation mechanisms into a unified battery model.

\section{Battery Mechanisms and State Variables}

Following our proposal, our model captures three main mechanisms: thermal effects, transient SoC dynamics, and long-term degradation via SEI growth.:contentReference[oaicite:5]{index=5} We represent the battery by a state vector
\begin{equation}
  y(t) =
  \begin{bmatrix}
    V(t) \\ I(t) \\ R(t) \\ T(t) \\ \mathrm{SOC}(t) \\ \mathrm{SEI}(t)
  \end{bmatrix},
\end{equation}
where:
\begin{itemize}
  \item $V$ [V]: terminal or open-circuit voltage of the cell,
  \item $I$ [A]: current into the battery (sign convention chosen for charging),
  \item $R$ [$\Omega$]: effective internal resistance,
  \item $T$ [K]: cell temperature (assumed spatially uniform),
  \item $\mathrm{SOC}\in[0,1]$: state-of-charge (fraction of nominal capacity),
  \item $\mathrm{SEI}$: SEI layer ``thickness'' or a proxy for degradation (arbitrary units).
\end{itemize}

Our time evolution is described by a system of ODEs
\begin{equation}
  \frac{dy}{dt} = f(y, t; V_{\mathrm{source}}, \theta),
\end{equation}
where $V_{\mathrm{source}}(t)$ is the applied (policy-dependent) source voltage, and $\theta$ is the collection of model parameters (thermal, SEI, capacity, etc.). The right-hand side is built additively:
\begin{equation}
  f(y, t; V_{\mathrm{source}}) = \sum_{\text{mechanisms } m} f^{(m)}(y, t; V_{\mathrm{source}}),
\end{equation}
mirroring the modular fire spread formulation where multiple submodels contribute to the dynamics.:contentReference[oaicite:6]{index=6}

\subsection{Mechanism Modules in Code}

Each mechanism is implemented as a Python class with a method
\begin{center}
\verb|get_gradient(y, t, v_source)|
\end{center}
that returns the contribution to the state derivative in the packed form
\[
  \frac{dy}{dt}^{(m)} =
  \texttt{pack\_state}(
    \dot{V}^{(m)},\ \dot{I}^{(m)},\ \dot{R}^{(m)},\ \dot{T}^{(m)},\ \dot{\mathrm{SOC}}^{(m)},\ \dot{\mathrm{SEI}}^{(m)}).
\]
The full derivative is then the sum over all mechanisms.

In the current implementation, the active mechanisms are:
\begin{itemize}
  \item \textbf{Charging} (SoC dynamics),
  \item \textbf{Thermo} (thermal dynamics),
  \item (optionally) \textbf{SEI} (degradation, in either detailed or simplified form).
\end{itemize}

An additional \textbf{UpdateState} module applies algebraic updates at each time step to keep $V$, $R$, and $I$ consistent with SoC, temperature, and $V_{\mathrm{source}}$.

\section{Mathematical Model: Core ODEs}

In this section we derive the explicit equations for each mechanism based on the Python code. This forms the ``Mathematical Model'' section in the MAT292 rubric.:contentReference[oaicite:7]{index=7}

\subsection{State-of-Charge (Charging Mechanism)}

The charging mechanism is based on the standard Coulomb-counting relation:
\begin{equation}
  \frac{d}{dt} Q(t) = I(t),
\end{equation}
where $Q$ is charge in Coulombs. Defining
\begin{equation}
  \mathrm{SOC}(t) = \frac{Q(t)}{Q_{\mathrm{nominal}}}, \qquad
  Q_{\mathrm{nominal}} = C_{\mathrm{nominal}} \cdot 3600 \, ,
\end{equation}
with $C_{\mathrm{nominal}}$ the nominal capacity in ampere-hours [Ah], we differentiate:
\begin{align}
  \frac{d}{dt}\mathrm{SOC}(t)
    &= \frac{1}{Q_{\mathrm{nominal}}} \frac{dQ}{dt}
     = \frac{1}{C_{\mathrm{nominal}}\cdot 3600} I(t).
\end{align}

Therefore, the SoC ODE is
\begin{equation}
  \boxed{
  \frac{d\,\mathrm{SOC}}{dt} = \frac{I}{C_{\mathrm{nominal}}\cdot 3600}
  }
\end{equation}
and the charging mechanism contributes
\begin{equation}
  \dot{V}^{(\mathrm{chg})} = 0,\quad
  \dot{I}^{(\mathrm{chg})} = 0,\quad
  \dot{R}^{(\mathrm{chg})} = 0,\quad
  \dot{T}^{(\mathrm{chg})} = 0,\quad
  \dot{\mathrm{SEI}}^{(\mathrm{chg})} = 0.
\end{equation}

\subsection{Thermal Mechanism}

The thermal mechanism accounts for:
\begin{itemize}
  \item Ohmic (Joule) heating,
  \item Overpotential heating (difference between $V_{\mathrm{source}}$ and $V$),
  \item Cooling via Newton's law of cooling.
\end{itemize}
Let $m$ be the mass [kg], $c$ the specific heat [J/(kg\,K)], $k$ the cooling constant [1/s], and $T_{\mathrm{amb}}$ the ambient temperature [K]. The net temperature ODE is
\begin{equation}
  \frac{dT}{dt} = \left(\frac{dT}{dt}\right)_{\text{ohmic}}
                 + \left(\frac{dT}{dt}\right)_{\text{overpot}}
                 + \left(\frac{dT}{dt}\right)_{\text{cool}}.
\end{equation}

\subsubsection*{Ohmic Heating}

Resistive heating in the internal resistance $R$ produces heat at rate $P_{\text{ohm}} = I^2 R$ [W]. Distributing this heat over the thermal mass $mc$ gives
\begin{equation}
  \left(\frac{dT}{dt}\right)_{\text{ohmic}} = \frac{I^2 R}{m c}.
\end{equation}

\subsubsection*{Overpotential Heating}

The difference between the source voltage and terminal voltage,
\begin{equation}
  \Delta V = V_{\mathrm{source}} - V,
\end{equation}
corresponds to additional electrochemical overpotential and other losses. The power associated with this overpotential is $P_{\text{overpot}} = I(V_{\mathrm{source}} - V)$, leading to
\begin{equation}
  \left(\frac{dT}{dt}\right)_{\text{overpot}} =
  \frac{I(V_{\mathrm{source}} - V)}{m c}.
\end{equation}

\subsubsection*{Cooling to Ambient}

Assuming linear cooling,
\begin{equation}
  \left(\frac{dT}{dt}\right)_{\text{cool}} = -k\,(T - T_{\mathrm{amb}}).
\end{equation}

\subsubsection*{Total Temperature ODE}

Combining all three contributions:
\begin{equation}
  \boxed{
  \frac{dT}{dt} =
  \frac{I^2 R}{m c}
  + \frac{I(V_{\mathrm{source}} - V)}{m c}
  - k\,(T - T_{\mathrm{amb}})
  }
\end{equation}
with
\begin{equation}
  \dot{V}^{(\mathrm{thermo})} = 0,\quad
  \dot{I}^{(\mathrm{thermo})} = 0,\quad
  \dot{R}^{(\mathrm{thermo})} = 0,\quad
  \dot{\mathrm{SOC}}^{(\mathrm{thermo})} = 0,\quad
  \dot{\mathrm{SEI}}^{(\mathrm{thermo})} = 0.
\end{equation}

\subsection{SEI Growth Model}

Long-term degradation is represented by growth of the SEI layer on the anode. We implement two variants: a detailed stress-dependent model and a simplified version. Both follow an Arrhenius temperature dependence, consistent with literature on SEI kinetics.:contentReference[oaicite:8]{index=8}

\subsubsection{Arrhenius Factor}

Let $E_a$ be the activation energy [J/mol], $R_{\mathrm{gas}}$ the universal gas constant [J/(mol\,K)], and $k_0$ a pre-exponential factor. The temperature sensitivity is
\begin{equation}
  f_T(T) = \exp\!\left(-\frac{E_a}{R_{\mathrm{gas}}\, T}\right), \qquad
  k_T = k_0 f_T(T) = k_0 \exp\!\left(-\frac{E_a}{R_{\mathrm{gas}}\,T}\right).
\end{equation}

\subsubsection{Detailed Stress Function}

Let $U$ be an anode potential-like quantity. If an explicit open-circuit voltage $U_{\mathrm{ocv}}$ is provided, we set $U=U_{\mathrm{ocv}}$; otherwise we approximate
\begin{equation}
  U \approx U_{\mathrm{ref}}\,\mathrm{SOC},
\end{equation}
where $U_{\mathrm{ref}}$ is a reference potential. The stress factor has two parts:
\begin{align}
  \text{exponent} &= \gamma_{\mathrm{soc}} (U_{\mathrm{ref}} - U), \\
  \text{current\_term} &= \left(1 + \beta_I |I|\right)^{\nu},
\end{align}
and the full multiplier
\begin{equation}
  f_{QI}(\mathrm{SOC}, I, U)
  = A \exp\!\left(\gamma_{\mathrm{soc}} (U_{\mathrm{ref}} - U)\right)
    \left(1 + \beta_I |I|\right)^{\nu}.
\end{equation}

\subsubsection{Detailed SEI ODE}

The SEI growth rate is then
\begin{equation}
  \boxed{
  \frac{d\,\mathrm{SEI}}{dt}
  = k_T \, f_{QI}(\mathrm{SOC}, I, U)
  = k_0 \exp\!\left(-\frac{E_a}{R_{\mathrm{gas}}\,T}\right)\,
    A \exp\!\left(\gamma_{\mathrm{soc}} (U_{\mathrm{ref}} - U)\right)
    \left(1 + \beta_I |I|\right)^{\nu}
  }
\end{equation}
with all other derivatives from the SEI module set to zero.

\subsubsection{Simplified SEI Model}

The simplified version replaces the more complex potential dependence by a direct SoC and current dependence:
\begin{align}
  \text{soc\_factor} &= \max(0,\min(1,\mathrm{SOC})), \\
  \text{current\_factor} &= |I|, \\
  f_{QI}^{\mathrm{simp}}(\mathrm{SOC}, I)
    &= (1 + 2\,\text{soc\_factor}^2)(1 + 0.1\,\text{current\_factor}).
\end{align}
The simplified SEI ODE is
\begin{equation}
  \boxed{
  \frac{d\,\mathrm{SEI}}{dt}
  = k_0 \exp\!\left(-\frac{E_a}{R_{\mathrm{gas}}\,T}\right)
    (1 + 2\,\mathrm{SOC}^2)(1 + 0.1\,|I|)
  }.
\end{equation}

\subsection{Algebraic State Updates}

In addition to the differential equations, an \texttt{UpdateState} module enforces algebraic consistency at each time step.

\subsubsection{Current Update}

Given $V_{\mathrm{source}}$, terminal voltage $V$, and resistance $R$, Ohm's law gives
\begin{equation}
  I_{\mathrm{new}} = \frac{V_{\mathrm{source}} - V}{R}.
\end{equation}

\subsubsection{Temperature-Dependent Resistance}

We model resistance drift with a temperature-activated law:
\begin{equation}
  R_{\mathrm{new}} =
  R_0 \exp\left(
    \frac{E_a^{(R)}}{k^{(R)}}\left(\frac{1}{T} - \frac{1}{T_{\mathrm{ref}}}\right)
  \right),
\end{equation}
where:
\begin{itemize}
  \item $R_0$ is the nominal resistance at temperature $T_{\mathrm{ref}}$,
  \item $E_a^{(R)}$ is an effective activation energy (in eV),
  \item $k^{(R)}$ plays the role of a ``gas constant'' in eV/K (taken as 8.314 in code),
  \item $T_{\mathrm{ref}} = 298$ K is the reference temperature.
\end{itemize}

\subsubsection{OCV from SoC}

We map SoC to a per-cell open-circuit voltage using a piecewise linear interpolation over tabulated data for LiFePO$_4$ cells. Given data pairs $(s_i, V_i)$ with $s_i$ in percent, the OCV at a given $s \in [0,100]$ is
\begin{equation}
  \mathrm{OCV}(s)
  = V_{\mathrm{low}}
  + \frac{s - s_{\mathrm{low}}}{s_{\mathrm{high}} - s_{\mathrm{low}}}
    (V_{\mathrm{high}} - V_{\mathrm{low}}),
\end{equation}
where $s_{\mathrm{low}} \le s \le s_{\mathrm{high}}$ and $(s_{\mathrm{low}},V_{\mathrm{low}})$, $(s_{\mathrm{high}},V_{\mathrm{high}})$ are adjacent data points bracketing $s$.

In implementation,
\begin{equation}
  V_{\mathrm{new}} = \mathrm{OCV}(100\cdot \mathrm{SOC}).
\end{equation}

\subsubsection{Updated State Map}

The algebraic update maps an old state $y$ to a new state $y^{+}$:
\begin{equation}
  y^{+} =
  \begin{bmatrix}
    \mathrm{OCV}(100\cdot \mathrm{SOC}) \\
    \dfrac{V_{\mathrm{source}} - V}{R} \\
    R_0 \exp\left(
      \dfrac{E_a^{(R)}}{k^{(R)}}\left(\dfrac{1}{T} - \dfrac{1}{T_{\mathrm{ref}}}\right)
    \right) \\
    T \\
    \mathrm{SOC} \\
    \mathrm{SEI}
  \end{bmatrix}.
\end{equation}

\section{Charging Policies (Control Laws)}

The applied source voltage $V_{\mathrm{source}}(t)$ is determined by a \emph{charging policy}, implemented as a class with a method \verb|get_voltage(t, y)|. This is analogous to using different wind profiles or control strategies in the fire spread model.:contentReference[oaicite:9]{index=9}

\subsection{Constant Voltage (CV)}

A constant-voltage policy holds the source at a fixed value:
\begin{equation}
  V_{\mathrm{source}}(t) = V_{\mathrm{set}}.
\end{equation}
In code, this is implemented by a \texttt{CV} class with parameter \texttt{voltage}.

\subsection{Constant Current (CC)}

An ideal constant-current policy chooses $V_{\mathrm{source}}$ such that, in steady state,
$I \approx I_{\mathrm{set}}$. Using Ohm's law:
\begin{equation}
  V_{\mathrm{source}}(t) = \frac{I_{\mathrm{set}}}{R(t)}.
\end{equation}
This is implemented by a \texttt{CC} class with parameter \texttt{current = $I_{\mathrm{set}}$}.

\subsection{Pulsed Charging}

For pulsed charging, we alternate between a high-current phase and a rest phase. Let $I_{\mathrm{set}}$ be the pulse current, $t_{\mathrm{on}}$ the pulse duration, $t_{\mathrm{off}}$ the rest duration, and
\begin{equation}
  t_{\mathrm{cycle}} = t_{\mathrm{on}} + t_{\mathrm{off}}.
\end{equation}
Then
\begin{equation}
  V_{\mathrm{source}}(t) =
  \begin{cases}
    \dfrac{I_{\mathrm{set}}}{R(t)}, & \text{if } t \bmod t_{\mathrm{cycle}} < t_{\mathrm{on}}, \\[6pt]
    0, & \text{otherwise}.
  \end{cases}
\end{equation}

\subsection{Sinusoidal Charging}

For sinusoidal charging, we modulate the current with a sinusoid of amplitude $I_{\mathrm{amp}}$ and frequency $f$:
\begin{align}
  I_{\mathrm{target}}(t) &= I_{\mathrm{amp}}\sin(2\pi f t), \\
  V_{\mathrm{source}}(t) &= \frac{I_{\mathrm{target}}(t)}{R(t)}.
\end{align}

\subsection{Planned Extensions: CC--CV and Negative Pulses}

Our proposal also mentions:
\begin{itemize}
  \item CC--CV fast charging: a high constant current until a voltage cutoff, followed by a constant-voltage phase.:contentReference[oaicite:10]{index=10}
  \item Negative pulses or rest periods: brief discharge pulses or reverse currents to mitigate polarization and side reactions.
\end{itemize}
These can be implemented as additional policies that switch between different functional forms of $V_{\mathrm{source}}(t)$ based on SoC or voltage thresholds.

\section{Numerical Methodology}

\subsection{Time Integration: RK4}

Given the ODE system
\begin{equation}
  \frac{dy}{dt} = f(y, t; V_{\mathrm{source}}),
\end{equation}
we adopt the classical fourth-order Runge--Kutta (RK4) scheme. For a time step $\Delta t$ and current state $y_n \approx y(t_n)$:
\begin{align}
  k_1 &= f(y_n, t_n; V_{\mathrm{source}}(t_n)), \\
  k_2 &= f\left(y_n + \frac{\Delta t}{2}k_1, t_n + \frac{\Delta t}{2}; V_{\mathrm{source}}(t_n + \Delta t/2)\right), \\
  k_3 &= f\left(y_n + \frac{\Delta t}{2}k_2, t_n + \frac{\Delta t}{2}; V_{\mathrm{source}}(t_n + \Delta t/2)\right), \\
  k_4 &= f\left(y_n + \Delta t\,k_3, t_n + \Delta t; V_{\mathrm{source}}(t_n + \Delta t)\right),
\end{align}
and the update
\begin{equation}
  y_{n+1} = y_n + \frac{\Delta t}{6}\left(k_1 + 2k_2 + 2k_3 + k_4\right).
\end{equation}

In the code, this is implemented by \verb|rk4_step|, with a fixed time step $dt = 0.01$~s.

\subsection{Simulation Loop}

For a given charging policy, a single charging cycle is simulated as follows:
\begin{enumerate}
  \item Initialize $t=0$ and
    \[
      y_0 = \texttt{pack\_state}\big(
        V_0,\ I_0,\ R_0,\ T_0,\mathrm{SOC}_0,\mathrm{SEI}_0
      \big),
    \]
    with $\mathrm{SOC}_0 = 0$ and $\mathrm{SEI}_0$ given.
  \item While $\mathrm{SOC} < 1$:
    \begin{enumerate}
      \item Compute $V_{\mathrm{source}} = \texttt{policy.get\_voltage}(t,y)$.
      \item Apply algebraic update $y \leftarrow \texttt{update\_y}(y, V_{\mathrm{source}})$.
      \item Compute $y \leftarrow \texttt{rk4\_step}(y, t, dt, V_{\mathrm{source}})$.
      \item Increment $t \leftarrow t + dt$, log $(t, y)$.
    \end{enumerate}
\end{enumerate}

Multiple charging cycles can be simulated to study SEI growth over repeated use.

\section{Planned Analyses and Metrics}

To answer our modelling questions, we will compute several metrics from the simulation logs.

\subsection{Charging Speed}

\begin{itemize}
  \item \textbf{Metric}: Time $\tau_{\mathrm{full}}$ required for $\mathrm{SOC}$ to reach 1 starting from 0.
  \item \textbf{Comparison}: Evaluate $\tau_{\mathrm{full}}$ under different policies (CV, CC, pulse, sinusoidal, and later CC--CV).
\end{itemize}

\subsection{Charging Efficiency (Thermal Losses)}

Heat generation during a cycle is linked to the thermal ODE:
\begin{align}
  P_{\mathrm{loss}}(t)
    &= mc\,\frac{dT}{dt} + k mc (T - T_{\mathrm{amb}}) \\
    &= I^2 R + I(V_{\mathrm{source}} - V),
\end{align}
so that
\begin{equation}
  Q_{\mathrm{loss}} = \int_0^{\tau_{\mathrm{full}}} P_{\mathrm{loss}}(t)\,dt.
\end{equation}
We will approximate this integral numerically and use $Q_{\mathrm{loss}}$ as a measure of inefficiency.

\subsection{Degradation and Lifetime Proxy}

Because SEI growth permanently consumes cyclable lithium, we use the total growth
\begin{equation}
  \Delta \mathrm{SEI} = \mathrm{SEI}(\tau_{\mathrm{full}}) - \mathrm{SEI}(0)
\end{equation}
as a proxy for degradation per cycle. Over many cycles, this could be linked to capacity fade, following literature on SEI-based lifetime modelling.:contentReference[oaicite:11]{index=11}

\subsection{Sensitivity and Parameter Studies}

Inspired by the sensitivity analysis of wind speed and fuel parameters in the fire spread model,:contentReference[oaicite:12]{index=12} we plan to perform:
\begin{itemize}
  \item Temperature sensitivity: vary ambient $T_{\mathrm{amb}}$ and initial $T_0$.
  \item Resistance sensitivity: vary $R_0$ and activation parameters for resistance drift.
  \item Policy parameters: e.g., pulse durations, current amplitudes, and frequencies.
  \item SEI parameters: e.g., $k_0$ and $E_a$.
\end{itemize}
We will focus on how $\tau_{\mathrm{full}}$, $Q_{\mathrm{loss}}$, and $\Delta \mathrm{SEI}$ respond to these perturbations.

\section{Project Plan and Deliverables}

The MAT292 project overview specifies a proposal, weekly check-ins, a final report, and code deliverables.:contentReference[oaicite:13]{index=13} Our project aligns with these as follows:
\begin{itemize}
  \item \textbf{Proposal}: Submitted 2-page document summarizing motivation, mechanisms, and objectives.:contentReference[oaicite:14]{index=14}
  \item \textbf{Weekly progress}: Implement and refine code modules, run preliminary simulations, and add features (e.g., CC--CV, negative pulses).
  \item \textbf{Final report}: Structured like the fire spread model example --- including background, model formulation, analysis, and discussion.:contentReference[oaicite:15]{index=15}
  \item \textbf{Code}: Fully commented Python package, able to reproduce key plots from the report.
\end{itemize}

\section{Discussion and Future Work}

Our current model is deliberately reduced-order: it treats the battery as a lumped element with a single temperature and scalar SoC, and it approximates complex electrochemical processes by a few phenomenological laws. This is analogous to using an elliptical growth model for fire spread, which captures essential geometry but omits detailed fluid dynamics.

Future extensions could include:
\begin{itemize}
  \item Spatially resolved temperature or SoC (e.g., radial diffusion within electrodes),
  \item More detailed electrochemical kinetics (e.g., Butler--Volmer equations for interfacial currents),
  \item Coupling to a discharge model to simulate full charge--discharge cycles,
  \item Data-driven parameter estimation using experimental or literature data.
\end{itemize}

Despite these limitations, the current framework already allows meaningful comparisons of charging strategies and highlights trade-offs between speed, efficiency, and degradation.

\section{Conclusion}

We have formulated a modular ODE-based model for lithium-ion battery charging, incorporating:
\begin{itemize}
  \item thermal dynamics with ohmic and overpotential heating plus cooling,
  \item SoC evolution based on current and nominal capacity,
  \item SEI growth via Arrhenius-type kinetics with stress dependence,
  \item and multiple charging policies that shape the source voltage.
\end{itemize}

Our code implements these mechanisms in a structured way that mirrors the decomposition used in the fire spread model: each physical effect contributes a component to the state derivative, and numerical integration ties them together. This sets the stage for systematic comparison of charging methods and for deeper mathematical analysis of stability, sensitivity, and optimization in battery systems.

\vspace{1em}
\noindent\textbf{Figure placeholders:} \\
In the final report we plan to include figures such as:
\begin{itemize}
  \item Time evolution of $T$, $\mathrm{SOC}$, and $\mathrm{SEI}$ under different charging policies.
  \item Phase-plane plots (e.g., $T$ vs.\ $\mathrm{SOC}$).
  \item Parameter sensitivity plots (e.g., $\tau_{\mathrm{full}}$ vs.\ pulse frequency).
\end{itemize}
These can be inserted as
\begin{verbatim}
\begin{figure}[h]
  \centering
  % \includegraphics[width=0.7\textwidth]{fig_temperature_evolution.pdf}
  \caption{Example placeholder caption.}
  \label{fig:temp_evolution}
\end{figure}
\end{verbatim}

\begin{thebibliography}{99}

% These correspond to the references already collected in the proposal.:contentReference[oaicite:17]{index=17}

\bibitem{IEA_EV}
IEA,
\newblock ``Outlook for battery and energy demand -- global EV outlook 2024 -- analysis,''
\newblock 2024.

\bibitem{IEA_Batteries}
IEA,
\newblock ``Status of battery demand and supply -- batteries and secure energy transitions -- analysis.''

\bibitem{Thicke2022}
M.~Thicke,
\newblock ``How to extend battery life with optimised charging,''
\newblock Outbax, 2022.

\bibitem{Stellantis2025}
G.~Guillaume,
\newblock ``Stellantis unveils lighter, faster-charging EV battery,''
\newblock Reuters, 2025.

\bibitem{Meng2023}
D.~Meng, X.~Wang, M.~Chen, and J.~Wang,
\newblock ``Effects of environmental temperature on the thermal runaway of lithium-ion batteries during charging process,''
\newblock \emph{Journal of Loss Prevention in the Process Industries}, 2023.

\bibitem{MaciasBV}
G.~Macias,
\newblock ``Understanding the Butler--Volmer equation,'' 2023.

\bibitem{Wang2018}
A.~Wang, S.~Kadam, H.~Li, S.~Shi, and Y.~Qi,
\newblock ``Review on modeling of the anode solid electrolyte interphase (SEI) for lithium-ion batteries,''
\newblock \emph{npj Computational Materials}, 2018.

\bibitem{Tomaszewska2019}
A.~Tomaszewska et al.,
\newblock ``Lithium-ion battery fast charging: A review,''
\newblock \emph{eTransportation}, 2019.

\bibitem{Vermeer2021}
W.~Vermeer et al.,
\newblock ``A critical review on the effects of pulse charging of Li-ion batteries,'' 2021.

\bibitem{Huang2024}
Y.~Huang et al.,
\newblock ``Investigation of lithium-ion battery negative pulsed charging strategy using NSGA-II,''
\newblock \emph{Electronics}, 2024.

\bibitem{Althurthi2024}
S.~B.~Althurthi, K.~Rajashekara, and T.~Debnath,
\newblock ``Comparison of EV fast charging protocols and impact of sinusoidal half-wave fast charging methods on lithium-ion cells,''
\newblock \emph{World Electric Vehicle Journal}, 2024.

\end{thebibliography}

\end{document}
